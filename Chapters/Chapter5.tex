% Chapter Template

\chapter{Results for 1-D Studies} % Main chapter title

\label{Chapter5}

\lhead{Chapter 5. \emph{Results for 1-D Studies}} 

The goal of this work was to reduce the number of energy degrees of freedom in 
a given problem to a manageable size without compromising the accuracy of the 
model.  
However, the best expansions of a function can only be performed when the 
function is completely
known, which, in this case, means the full solution is required before the 
expansion, which precludes the need for an expansion.  As such, 
this work focused on two classes of basis expansions.  The first class is the 
expansions based on the test problem of interest, which provide insight to the 
best that a basis set can do.  This class is compared to the sets of basis 
functions that are based on small, yet similar models to the test problem.  
These small models are quick to solve, and hopefully similar enough to the 
test problem that an expansion based on the model would be accurate enough for 
the larger test problem.  Thus, another focus of this work was to identify how 
similar a model had to be to a given test problem to provide effective basis 
functions for expansion.

Typically, reactor analysis attempts to compute pin fission densities (or 
powers) with sub-$1\%$ errors.  This work added an additional buffer, and 
focused on the minimization of the energy expansion order 
required to 
achieve sub-$0.1\%$ maximum relative errors in the fission density. The 
reference solution in all cases was a full multi-group response matrix solution 
for the given test problem with consistent angular expansion used throughout.  
This consistency ensured the observed changes in the solution were functions of only the 
energy basis used.  In practice, it is 
not possible to truly separate the effects of space, angle, and energy because 
the responses in each phase-space variable are coupled, but it is assumed that 
this treatment will provide adequate insight into the effect of the chosen 
energy expansion.

With the exception of `DLP' and `mDLP', each curve was generated using a KLT 
basis with distinct snapshot data, as described previously in Chapter 
\ref{Chapter4}. 
The mDLP results represent Comparison of mDLP applied to the 10-pin problem using the 
            238-group cross-section library.the best case previously observed by 
\citet{Roberts2014}, which was to use mDLP-1 with the average flux profile from 
the complete test problem of interest as the shape function.  Therefore, the 
mDLP 
results are not practical because the solution for the test problem must be 
known {\it a priori}.  

% Because the test problems and snapshot models studied were relatively small, 
% timing studies would provide little insight.  However, for large 2-D or 3-D 
% models (e.g., assemblies or full cores), the snapshot models (e.g., pincells 
% or sub-assemblies) should be orders of magnitude less computationally expensive
% than the full model of interest. 

\section{mDLP Comparison}

\FIGURE{fig:mDLP_44} shows a comparison between practical applications of 
both versions of mDLP when applied to the 10-pin problem using the 44-group 
cross-section library.  \FIGURE{fig:mDLP_238} shows the same comparison, 
but instead for the 238-group cross-section library.  In 
\FIG{fig:mDLP_238}, both plots show 
the 
same data, but the left plot is truncated to order 43 to ease comparison to the 
44-group data. For these figures, the denotation ``full'' means the shape 
vector was the flux spectrum averaged over space across all 10 pins.  The 
denotation ``UO$_2$'' used the flux spectrum from a UO$_2$ pin as the 
shape vector.  Likewise, the denotation ``MOX'' used the shape vector as the 
flux spectrum from a MOX pin.

\begin{figure*}[tb]
    \centering
    \includegraphics[trim=.1cm .25cm 2.0cm .4cm clip=true, 
    totalheight=0.28\textheight]{Figures/c/10-pin/44/rf_plots/phi/%
        mDLP_comparison_fission-44}
    \caption{Comparison of mDLP applied to the 10-pin problem using 44-group 
        cross-section library.}
    \label{fig:mDLP_44}
\end{figure*}

\begin{figure*}[tb]
    \centering
    \begin{subfigure}{0.5\textwidth}
        \centering
        \includegraphics[trim=.1cm .25cm 2.0cm .4cm clip=true, 
    totalheight=0.238\textheight]{Figures/c/10-pin/238/rf_plots/phi/%
        mDLP_comparison_fission-44}
    \end{subfigure}%
    \begin{subfigure}{0.5\textwidth}
        \centering
        \includegraphics[trim=.1cm .25cm 2.0cm .4cm clip=true, 
    totalheight=0.238\textheight]{Figures/c/10-pin/238/rf_plots/phi/%
        mDLP_comparison_fission-238}
    \end{subfigure}
    \caption{Comparison of mDLP applied to the 10-pin problem using the 
        238-group cross-section library.}
    \label{fig:mDLP_238}
\end{figure*}

All cases of mDLP-2 performed worse than to mDLP-1.  The ``full'' denotation 
represents the impractical expansion in that the complete problem is required 
prior to the expansion. The ``UO$_2$'' and ``MOX'' denotations 
represent more practical expansions as they each require only solving a 
pincell 
problem to generate the expansion; however, these two models do not have the 
physics for one of the fuel types, and thus do not perform as well.  As expected, 
the ``full'' expansions of 
mDLP-1 perform the best in general and were chosen as the comparison 
point for the rest of this work.  Lastly, the increased error at high-order of 
\FIG{fig:mDLP_238} is due to non-orthogonality of the high-order basis 
functions due to accumulating roundoff error during orthogonalization.

\section{Energy Spectra for the Test Problems}

The 10-pin test problem was first solved to generate flux profiles to be used 
for mDLP and KLT expansions.  In \FIG{fig:spectrum-44}, the spatially 
averaged flux spectra are presented for the 10-pin problem with the 44-group 
cross-section library.  These averaged spectra were used as the shape vectors for 
mDLP, while 
the non spatially-averaged data were used for KLT expansion.  These spectra can 
be compared to those in \FIG{fig:spectrum-238}, which shows the resulting 
spectra for the 10-pin problem using the 238-group cross-section library.  As 
can be observed in each of these figures, the basic shape of the spectrum are 
nearly the same, with the 10-pin solution lying as average between the two 
individual pin spectra, UO$_2$ and MOX.

\begin{figure*}[tb]
    \centering
    \includegraphics[trim=.1cm .25cm 2.0cm .4cm clip=true, 
    totalheight=0.28\textheight]{Figures/c/10-pin/44/rf_plots/%
        44group_spectra_energy}
    \caption{Flux spectrum for the 10-pin problem using 44-group cross-section 
        library}
    \label{fig:spectrum-44}
\end{figure*}

%\begin{figure*}[tb]
%    \centering
%    \includegraphics[trim=.1cm .25cm 2.0cm .4cm clip=true, 
%    totalheight=0.28\textheight]{Figures/c/10-pin/44/reference_figures/%
%        44monent0_spectra_energy}
%\end{figure*}

\begin{figure*}[tb]
    \centering
    \includegraphics[trim=.1cm .25cm 2.0cm .4cm clip=true, 
    totalheight=0.28\textheight]{Figures/c/10-pin/238/rf_plots/%
        238group_spectra_energy}
    \caption{Flux spectrum for the 10-pin problem using 238-group cross-section 
        library}
    \label{fig:spectrum-238}
\end{figure*}

%\begin{figure*}[tb]
%    \centering
%    \includegraphics[trim=.1cm .25cm 2.0cm .4cm clip=true, 
%    totalheight=0.28\textheight]{Figures/c/10-pin/238/reference_figures/%
%        238monent0_spectra_energy}
%\end{figure*}

\FIGURE{fig:spectrum0-238} shows the spatially averaged spectrum for Core 0 
of the BWR test problem, where the spectra denoted ``assay1'' or 
``assay2'' are the spatially averaged energy spectra for each assembly 
in Core-0 of the BWR test problem.  It is apparent that these spectra 
do not differ as much as the spectra associated with the 10-pin 
problem.  As will be shown later, this 
leads to improved performance of mDLP because there is not as much 
variation between the various parts in the model.  The spectra for Core-1, 
shown in \FIG{fig:spectrum1-238}, is similar to that of Core-0.  However, 
the spectra difference becomes more 
pronounced when observing the spectra for Core-2, shown in Fig. 
\ref{fig:spectrum2-238}.

\begin{figure*}[tb]
    \centering
    \includegraphics[trim=.1cm .25cm 2.0cm .4cm clip=true, 
    totalheight=0.28\textheight]{Figures/c/bwrcore/238/0/rf_plots/%
        238group_spectra_energy}
    \caption{Flux spectrum for the BWR-Core 0 problem using 238-group 
        cross-section library}
    \label{fig:spectrum0-238}
\end{figure*}

\begin{figure*}[tb]
    \centering
    \includegraphics[trim=.1cm .25cm 2.0cm .4cm clip=true, 
    totalheight=0.28\textheight]{Figures/c/bwrcore/238/1/rf_plots/%
        238group_spectra_energy}
    \caption{Flux spectrum for the BWR-Core 1 problem using 238-group 
        cross-section library}
    \label{fig:spectrum1-238}
\end{figure*}

\begin{figure*}[tb]
    \centering
    \includegraphics[trim=.1cm .25cm 2.0cm .4cm clip=true, 
    totalheight=0.28\textheight]{Figures/c/bwrcore/238/2/rf_plots/%
        238group_spectra_energy}
    \caption{Flux spectrum for the BWR-Core 2 problem using 238-group 
        cross-section library}
    \label{fig:spectrum2-238}
\end{figure*}

\section{10-Pin Test Problem}

As previously discussed, the KLT was applied to the 10-pin test problem using 
several snapshot models.  Additionally, there were several kinds of snapshots 
taken 
from each snapshot model i.e., scalar flux ($\phi$), leftward partial current 
($J_{\text{left}}$), and higher-order angular 
moments.  These snapshots were combined in various ways for the 10-pin problem 
and are presented in this section.  Each of the plots in this section are shown 
up to order $G-1$, where $G$ is the number of groups in the cross-section 
library.  The last order is omitted because orthogonal basis functions are 
exact to machine precision when using a complete expansion.  Some of the DLP and 
mDLP results will not appear to converge nearing the final order, which is due 
to non-orthogonality of the high orders from accumulating roundoff error in the 
orthogonalization.  Furthermore, since the problem is non-linear, the error is 
not guaranteed to decrease with increasing expansion order, however, the error 
should decrease on the average with increasing expansion order.

\subsection{44-Group Results}

\FIGURE{fig:10-pin-flux-only} shows the performance of various formulations 
of the KLT as applied to the 10-pin problem.  The snapshots used for the figure 
came from only the 44-group scalar flux.  When using snapshots from the full 
assembly model (10-pin), 
    the relative error fell below the 0.1$\%$ threshold at an energy order of 
    approximately 12, while more practical models, e.g., 
Combined-Pin, require approximately order 20 to reach the goal.  The results of 
the 
Combined-Pins model are far from 
the goal of using approximately five energy degrees of freedom to reach the 
error goal.  
Essentially all of the KLT formulations outperformed mDLP-1 with the 
exception of the 
individual pins (UO$_2$ and MOX), which is expected as those models lack 
physics 
for half the problem space.

\begin{figure*}[tb]
    \centering
    \includegraphics[trim=.1cm .25cm 2.0cm .4cm clip=true, 
    totalheight=0.28\textheight]{Figures/c/10-pin/44/rf_plots/phi/%
        energy_basis_comparison_fission-44}
    \caption{Performance of the KLT when applied to the 10-pin test problem 
        with snapshots of only $\phi$.}
    \label{fig:10-pin-flux-only}
\end{figure*}

\FIGURE{fig:10-pin-partial-only} shows the performance of the KLT using 
snapshots of only $J_{\text{left}}$ from the 44-group snapshot 
models. As shown, the best case KLT can reach the goal by approximately 
order 6, while more practical models, e.g., Combined-Pin, require approximately 
order 20 to 
reach the error goal.

\begin{figure*}[tb]
    \centering
    \includegraphics[trim=.1cm .25cm 2.0cm .4cm clip=true, 
    totalheight=0.28\textheight]{Figures/s/10-pin/44/rf_plots/partial/%
        partial_energy_basis_comparison_fission-44}
    \caption{Performance of the KLT when applied to the 10-pin test problem 
        with snapshots of only $J_{\text{left}}$.}
    \label{fig:10-pin-partial-only}
\end{figure*}

\FIGURE{fig:10-pin-combined} shows the results of using the combined 
snapshots of $\phi$ and $J_{\text{left}}$ each from the 
44-group models.  With this approach, the best case still reaches the goal at 
about order 6, but the practical case of Combined-Pins was improved to 
achieving the goal 
by 
order 15.  In general, including both sets of snapshots improves the 
performance 
because KLT can extract the most important information from both sets and 
create highly effective basis functions.

\begin{figure*}[tb]
    \centering
    \includegraphics[trim=.1cm .25cm 2.0cm .4cm clip=true, 
    totalheight=0.28\textheight]{Figures/c/10-pin/44/rf_plots/partial/%
        partial_energy_basis_comparison_fission-44}
    \caption{Performance of the KLT when applied to the 10-pin test problem 
        with snapshots of both $\phi$ and leftward partial current.}
    \label{fig:10-pin-combined}
\end{figure*}

\subsection{238-Group Results}

Similar to the 44-group section, the figures in this section will be presented 
to showcase how the choice of snapshots impacts the performance of the KLT 
applied 
to the 10-pin problem using the 238-group cross-section library. All of the 
figures in 
this section include side-by-side comparisons of the same data, one 
shown to 43th order, while the other presents the full spectrum up to 
237th order.  

\FIGURE{fig:10-pin-238phi} presents the results from using 
snapshots of only $\phi$ in the basis generation.  The best expansion 
requires approximately order 14, while the Combined-Pins case requires 
approximately order 24 to remain under the error goal.  The individual pin 
models do not perform well, as is expected, as each lacks the information from 
an 
entire type of fuel pin.

\begin{figure*}[tb]
    \centering
    \begin{subfigure}{0.5\textwidth}
        \centering
        \includegraphics[trim=.1cm .25cm 2.0cm .4cm clip=true, 
        totalheight=0.238\textheight]{Figures/c/10-pin/238/rf_plots/phi/%
            energy_basis_comparison_fission-44}
    \end{subfigure}%
    \begin{subfigure}{0.5\textwidth}
        \centering
        \includegraphics[trim=.1cm .25cm 2.0cm .4cm clip=true, 
        totalheight=0.238\textheight]{Figures/c/10-pin/238/rf_plots/phi/%
            energy_basis_comparison_fission-238}
    \end{subfigure}
    \caption{Relative error for 238-group, 10-pin test problem using snapshots 
    of only $\phi$}
    \label{fig:10-pin-238phi}
\end{figure*}

Presented in \FIG{fig:10-pin-238partial} are the results from using 
snapshots of only $J_{\text{left}}$ for basis generation. The 
impractical case of the 10-pin model requires approximately order 6 to reach 
the error goal, while the practical Combined-Pins model requires approximately 
order 22 to reach the error goal.  These results are an improvement for the 
10-pin model over using $\phi$ snapshots. However, the Combined-Pins model 
results are relatively unchanged as compared to using $\phi$ snapshots.

\begin{figure*}[tb]
    \centering
    \begin{subfigure}{0.5\textwidth}
        \centering
        \includegraphics[trim=.1cm .25cm 2.0cm .4cm clip=true, 
        totalheight=0.238\textheight]{Figures/s/10-pin/238/rf_plots/partial/%
            partial_energy_basis_comparison_fission-44}
    \end{subfigure}%
    \begin{subfigure}{0.5\textwidth}
        \centering
        \includegraphics[trim=.1cm .25cm 2.0cm .4cm clip=true, 
        totalheight=0.238\textheight]{Figures/s/10-pin/238/rf_plots/partial/%
            partial_energy_basis_comparison_fission-238}
    \end{subfigure}
    \caption{Relative error for 238-group, 10-pin test problem using snapshots 
        of only $J_{\text{left}}$}
    \label{fig:10-pin-238partial}
\end{figure*}

\FIGURE{fig:10-pin-238combined} shows the results from combining 
together the snapshots of both $\phi$ and $J_{\text{left}}$. In this case, the 
10-pin model requires approximately order 10 to 
reach the goal, while the Combined-Pins model requires approximately order 14 
to 
reach the goal.  As compared to the 44-group data, these data from 238-group do 
not improve to the same degree when $\phi$ and $J_{\text{left}}$ 
snapshots are combined.

\begin{figure*}[tb]
    \centering
    \begin{subfigure}{0.5\textwidth}
        \centering
        \includegraphics[trim=.1cm .25cm 2.0cm .4cm clip=true, 
        totalheight=0.238\textheight]{Figures/c/10-pin/238/rf_plots/partial/%
            partial_energy_basis_comparison_fission-44}
    \end{subfigure}%
    \begin{subfigure}{0.5\textwidth}
        \centering
        \includegraphics[trim=.1cm .25cm 2.0cm .4cm clip=true, 
        totalheight=0.238\textheight]{Figures/c/10-pin/238/rf_plots/partial/%
            partial_energy_basis_comparison_fission-238}
    \end{subfigure}
    \caption{Relative error for 238-group, 10-pin test problem using snapshots 
        of both $\phi$ and $J_{\text{left}}$}
    \label{fig:10-pin-238combined}
\end{figure*}

Note that when comparing the cases between the 44-group and 238-group results, 
basis functions from a given type of model (e.g., Combined Pin) perform 
similarly despite the number of groups.  It appears that for simple problems, 
KLT basis sets may contain a relatively constant amount of information despite increasing 
group size, 
but more effort is required to confirm this trend.

\section{BWR Test Problem}

The BWR test problem refers to the grouping of three 
different core configurations 
as discussed in Chapter \ref{Chapter4}.  Each of the configurations 
was used with both 44-group and 238-group cross-section libraries.  The 
configurations were designed with increasing inhomogeneity leading to more difficult 
models. The difficulty results in a larger error for the same energy order while progressing 
    through configurations. Thus, the performance of 
each expansion is expected to worsen with increasing configuration number.  

For each configuration, the Full-Core model is expected to perform the best, as 
it uses all available unique snapshots for basis generation, while the 
Combined-Assemblies and the Combined-Pins models represent practical cases for 
basis generation. 
 Note that for this test problem, mDLP and DLP perform quite well, which is 
due to the relatively small difference between the spatially averaged 
flux profile as discussed previously.  Again, the mDLP results presented here 
used the spatially averaged flux profile from the Full-Core model as the shape 
vector, and, thus, do not represent a practical performance of mDLP, but rather 
should be compared to the Full-Core KLT results.

\subsection{Configuration 0}

\subsubsection{44-Group Results}

As \FIG{fig:BWR0_phi} shows, it takes approximately order 6 for the 
Full-Core model to reach the goal of sub-0.1\% relative error in the fission 
density.  Whereas the practical Combined models require approximately order 15 
to 
reach the goal.  These results are when using snapshots of only $\phi$.  

\begin{figure*}[tb]
    \centering
    \includegraphics[trim=.1cm .25cm 2.0cm .4cm clip=true, 
    totalheight=0.28\textheight]{Figures/c/bwrcore/44/0/rf_plots/phi/%
        energy_basis_comparison_fission-44}
    \caption{Relative error for the 44-group, BWR-Core 0 test problem using 
        snapshots of only $\phi$}
    \label{fig:BWR0_phi}
\end{figure*}

Many of the results are improved if snapshots of $J_{\text{left}}$ are 
used for basis generation with the KLT, as shown in Fig. 
\ref{fig:BWR0_partial}.  The Full-Core model requires approximately order 4, 
while the Combined-Assemblies model requires approximately order 10 to reach 
the goal.

\begin{figure*}[tb]
    \centering
    \includegraphics[trim=.1cm .25cm 2.0cm .4cm clip=true, 
    totalheight=0.28\textheight]{Figures/s/bwrcore/44/0/rf_plots/partial/%
        partial_energy_basis_comparison_fission-44}
    \caption{Relative error for the 44-group, BWR-Core 0 test problem using 
        snapshots of only $J_{\text{left}}$}
    \label{fig:BWR0_partial}
\end{figure*}

The best results are obtained when snapshots of both $\phi$ and 
$J_{\text{left}}$ are combined together to generate the KLT basis.  
These results are presented in \FIG{fig:BWR0_combined}.  The Full-Core 
model requires order 4 to reach the error goal.  The Combined-Assemblies 
reached the goal at approximately order 10.  Finally, the Combined-Pins results 
are greatly improved, and require approximately order 13 to reach the goal when 
using both types of snapshots.

\begin{figure*}[tb]
    \centering
    \includegraphics[trim=.1cm .25cm 2.0cm .4cm clip=true, 
    totalheight=0.28\textheight]{Figures/c/bwrcore/44/0/rf_plots/partial/%
        partial_energy_basis_comparison_fission-44}
    \caption{Relative error for the 44-group, BWR-Core 0 test problem using 
        snapshots of both $\phi$ and $J_{\text{left}}$}
    \label{fig:BWR0_combined}
\end{figure*}

\subsubsection{238-Group Results}

As before, the results in this section are presented as side-by-side 
comparisons of the 238-group data.  The left side is shown to only order 43 to 
ease comparison to the 44-group data, while the right side shows up to 
237th order for the 238-group data.

\FIGURE{fig:BWR0_phi-238} shows the results from using snapshots of only the 
$\phi$ for the 238-group cross-section library.  The impractical model of 
Full-Core requires approximately order 12 to reach the goal, while the 
Combined-Assemblies case requires approximately order 21.  Each of these models 
also 
reach the second goal of an order of magnitude reduction in the required energy 
degrees of freedom.  The Combined-Pins model does not reach the error goal 
within the order goal.  

\begin{figure*}[tb]
    \centering
    \begin{subfigure}{0.5\textwidth}
        \centering
    \includegraphics[trim=.1cm .25cm 2.0cm .4cm clip=true, 
    totalheight=0.238\textheight]{Figures/c/bwrcore/238/0/rf_plots/phi/%
        energy_basis_comparison_fission-44}
    \end{subfigure}%
    \begin{subfigure}{0.5\textwidth}
        \centering
    \includegraphics[trim=.1cm .25cm 2.0cm .4cm clip=true, 
    totalheight=0.238\textheight]{Figures/c/bwrcore/238/0/rf_plots/phi/%
        energy_basis_comparison_fission-238}
    \end{subfigure}
    \caption{Relative error for 238-group, BWR-Core 0 test problem using 
        snapshots of only $\phi$}
    \label{fig:BWR0_phi-238}
\end{figure*}

The required orders for most cases is reduced when using snapshots of only 
$J_{\text{left}}$ to generate the basis as shown in Fig. 
\ref{fig:BWR0_partial-238}.  The Full-Core model requires approximately order 3 
to reach the goal, while the Combined-Assemblies model requires approximately 
order 12.  However, the performance of the Combined-Pins model is actually 
worse 
as compared to using $\phi$ snapshots.

\begin{figure*}[tb]
    \centering
    \begin{subfigure}{0.5\textwidth}
        \centering
    \includegraphics[trim=.1cm .25cm 2.0cm .4cm clip=true, 
    totalheight=0.238\textheight]{Figures/s/bwrcore/238/0/rf_plots/partial/%
        partial_energy_basis_comparison_fission-44}
    \end{subfigure}%
    \begin{subfigure}{0.5\textwidth}
        \centering
    \includegraphics[trim=.1cm .25cm 2.0cm .4cm clip=true, 
    totalheight=0.238\textheight]{Figures/s/bwrcore/238/0/rf_plots/partial/%
        partial_energy_basis_comparison_fission-238}
    \end{subfigure}
    \caption{Relative error for 238-group, BWR-Core 0 test problem using 
        snapshots of only $J_{\text{left}}$}
    \label{fig:BWR0_partial-238}
\end{figure*}

Once again, the best results are obtained by combining the snapshots of the 
$\phi$ and $J_{\text{left}}$ together as shown in Fig. 
\ref{fig:BWR0_combined-238}.  The required order for the Full-Core model 
remains at about 3, and the Combined-Assemblies model requires approximately 
order 14 to reach the error goal.  The Combined-Pins model is greatly improved 
and reaches the error goal at about order 20.  When using both types of 
snapshots, the Combined-Pins model reached the required order goal of an order 
of magnitude reduction.

\begin{figure*}[tb]
    \centering
    \begin{subfigure}{0.5\textwidth}
        \centering
    \includegraphics[trim=.1cm .25cm 2.0cm .4cm clip=true, 
    totalheight=0.238\textheight]{Figures/c/bwrcore/238/0/rf_plots/partial/%
        partial_energy_basis_comparison_fission-44}
    \end{subfigure}%
    \begin{subfigure}{0.5\textwidth}
        \centering
    \includegraphics[trim=.1cm .25cm 2.0cm .4cm clip=true, 
    totalheight=0.238\textheight]{Figures/c/bwrcore/238/0/rf_plots/partial/%
        partial_energy_basis_comparison_fission-238}
    \end{subfigure}
    \caption{Relative error for 238-group, BWR-Core 0 test problem using 
        snapshots of both $\phi$ and $J_{\text{left}}$}
    \label{fig:BWR0_combined-238}
\end{figure*}

\subsection{Configuration 1}

\subsubsection{44-Group Results}

\FIGURE{fig:BWR1_phi} shows that approximately order 7 is required for the 
Full-Core model to reach the error threshold.  However, the practical,  
Combined-Assemblies and Combined-Pins models require approximately order 17 and 
order 26 to reach the error threshold.  These results are when using snapshots 
of 
only $\phi$.  Both of the Combined models are far from the order goal 
of approximately 5 order required to meet the error threshold.

\begin{figure*}[tb]
    \centering
    \includegraphics[trim=.1cm .25cm 2.0cm .4cm clip=true, 
    totalheight=0.28\textheight]{Figures/c/bwrcore/44/1/rf_plots/phi/%
        energy_basis_comparison_fission-44}
    \caption{Relative error for the 44-group, BWR-Core 1 test problem using 
        snapshots of only $\phi$}
    \label{fig:BWR1_phi}
\end{figure*}

Many of the results are improved when snapshots of $J_{\text{left}}$ are used 
for basis generation with the KLT instead of $\phi$. 
 These results are shown in \FIG{fig:BWR1_partial}.  The Full-Core 
model requires approximately order 7, while the Combined-Assemblies model 
requires approximately order 21 to reach the error goal.  The Combined-Pins 
model 
require approximately order 27.  

\begin{figure*}[tb]
    \centering
    \includegraphics[trim=.1cm .25cm 2.0cm .4cm clip=true, 
    totalheight=0.28\textheight]{Figures/s/bwrcore/44/1/rf_plots/partial/%
        partial_energy_basis_comparison_fission-44}
    \caption{Relative error for the 44-group, BWR-Core 1 test problem using 
        snapshots of only $J_{\text{left}}$}
    \label{fig:BWR1_partial}
\end{figure*}

The best results are obtained when snapshots of both $\phi$ and 
$J_{\text{left}}$ are combined together to generate the KLT basis.  
These results are presented in \FIG{fig:BWR1_combined}.  The Full-Core 
model requires order 7 to reach the error goal.  Both the Combined-Assemblies 
and Combined-Pins models reached the goal at approximately order 19, when 
using both types of snapshots.

\begin{figure*}[tb]
    \centering
    \includegraphics[trim=.1cm .25cm 2.0cm .4cm clip=true, 
    totalheight=0.28\textheight]{Figures/c/bwrcore/44/1/rf_plots/partial/%
        partial_energy_basis_comparison_fission-44}
    \caption{Relative error for the 44-group, BWR-Core 1 test problem using 
        snapshots of both $\phi$ and $J_{\text{left}}$}
    \label{fig:BWR1_combined}
\end{figure*}

\subsubsection{238-Group Results}

\FIGURE{fig:BWR1_phi-238} shows the results from using snapshots of only 
$\phi$ for the 238-group cross-section library.  The impractical model of 
Full-Core requires approximately order 13 to reach the goal, while the 
Combined-Assemblies case requires approximately order 26.  Each of these models 
also 
reach the second goal of an order of magnitude reduction in the required energy 
degrees of freedom.  The Combined-Pins model does not reach the error goal 
within the order goal.  

\begin{figure*}[tb]
    \centering
    \begin{subfigure}{0.5\textwidth}
        \centering
    \includegraphics[trim=.1cm .25cm 2.0cm .4cm clip=true, 
    totalheight=0.238\textheight]{Figures/c/bwrcore/238/1/rf_plots/phi/%
        energy_basis_comparison_fission-44}
    \end{subfigure}%
    \begin{subfigure}{0.5\textwidth}
        \centering
    \includegraphics[trim=.1cm .25cm 2.0cm .4cm clip=true, 
    totalheight=0.238\textheight]{Figures/c/bwrcore/238/1/rf_plots/phi/%
        energy_basis_comparison_fission-238}
    \end{subfigure}
    \caption{Relative error for 238-group, BWR-Core 1 test problem using 
        snapshots of only $\phi$}
    \label{fig:BWR1_phi-238}
\end{figure*}

Again, the required orders for most cases is reduced when using snapshots of 
only $J_{\text{left}}$ as shown in \FIG{fig:BWR1_partial-238}.  
The Full-Core model requires approximately order 8 to reach the goal, while the 
Combined-Assemblies model requires approximately order 9.  The Combined-Pins 
model  reaches the error goal by approximately order 26.

\begin{figure*}[tb]
    \centering
    \begin{subfigure}{0.5\textwidth}
        \centering
    \includegraphics[trim=.1cm .25cm 2.0cm .4cm clip=true, 
    totalheight=0.238\textheight]{Figures/s/bwrcore/238/1/rf_plots/partial/%
        partial_energy_basis_comparison_fission-44}
    \end{subfigure}%
    \begin{subfigure}{0.5\textwidth}
        \centering
    \includegraphics[trim=.1cm .25cm 2.0cm .4cm clip=true, 
    totalheight=0.238\textheight]{Figures/s/bwrcore/238/1/rf_plots/partial/%
        partial_energy_basis_comparison_fission-238}
    \end{subfigure}
    \caption{Relative error for 238-group, BWR-Core 1 test problem using 
        snapshots of only $J_{\text{left}}$}
    \label{fig:BWR1_partial-238}
\end{figure*}

Combining the two types of snapshots together has little effect at low orders 
for this configuration as shown in \FIG{fig:BWR1_combined-238}.  The 
required order for the full core model remains at about 8, and the 
Combined-Assemblies model requires approximately order 15 to reach the error 
goal.  The 
Combined-Pins model is greatly improved and reaches the error goal at about 
order 
26.  It appears that for this configuration $\phi$ actually worsens the 
expansion.

\begin{figure*}[tb]
    \centering
    \begin{subfigure}{0.5\textwidth}
        \centering
    \includegraphics[trim=.1cm .25cm 2.0cm .4cm clip=true, 
    totalheight=0.238\textheight]{Figures/c/bwrcore/238/1/rf_plots/partial/%
        partial_energy_basis_comparison_fission-44}
    \end{subfigure}%
    \begin{subfigure}{0.5\textwidth}
        \centering
    \includegraphics[trim=.1cm .25cm 2.0cm .4cm clip=true, 
    totalheight=0.238\textheight]{Figures/c/bwrcore/238/1/rf_plots/partial/%
        partial_energy_basis_comparison_fission-238}
    \end{subfigure}
    \caption{Relative error for 238-group, BWR-Core 1 test problem using 
        snapshots of both $\phi$ and $J_{\text{left}}$}
    \label{fig:BWR1_combined-238}
\end{figure*}

\subsection{Configuration 2}

\subsubsection{44-Group Results}

The final configuration of the BWR test problem leads to results very similar to 
the previous two configurations, but requires slightly higher-order expansion to achieve the 
error goal, as expected.  As \FIG{fig:BWR2_phi} shows, it takes 
approximately order 12 for the Full-Core model to reach the goal of sub-0.1\% 
relative error in the fission density.  Whereas the practical Combined models 
take approximately order 18 to reach the goal.  These results are when using 
snapshots of only $\phi$.  The Combined-Pins model reaches the goal at 
approximately order 30.  None of these models can achieve the order goal of 
approximately 5 orders for the 44-group test problems.

\begin{figure*}[tb]
    \centering
    \includegraphics[trim=.1cm .25cm 2.0cm .4cm clip=true, 
    totalheight=0.28\textheight]{Figures/c/bwrcore/44/2/rf_plots/phi/%
        energy_basis_comparison_fission-44}
    \caption{Relative error for the 44-group, BWR-Core 2 test problem using 
        snapshots of only $\phi$}
    \label{fig:BWR2_phi}
\end{figure*}

Some of the results are slightly improved if instead snapshots of 
$J_{\text{left}}$ are used for basis generation with the KLT, as \FIG{fig:BWR2_partial} shows.  The 
Full-Core model requires approximately order 10, 
while the Combined-Assemblies model requires approximately order 24 to reach 
the goal, which is worse than the expansion using snapshots of $\phi$.

\begin{figure*}[tb]
    \centering
    \includegraphics[trim=.1cm .25cm 2.0cm .4cm clip=true, 
    totalheight=0.28\textheight]{Figures/s/bwrcore/44/2/rf_plots/partial/%
        partial_energy_basis_comparison_fission-44}
    \caption{Relative error for the 44-group, BWR-Core 2 test problem using 
        snapshots of only $J_{\text{left}}$}
    \label{fig:BWR2_partial}
\end{figure*}

The best expansions for the practical cases are obtained when snapshots of both 
$\phi$ and $J_{\text{left}}$ are combined together to generate 
the KLT basis.  These results are presented in \FIG{fig:BWR2_combined}.  
The Full-Core model requires order 12 to reach the error goal.  The 
Combined-Assemblies and the Combined-Pins models reached the goal at 
approximately order 
19.  

\begin{figure*}[tb]
    \centering
    \includegraphics[trim=.1cm .25cm 2.0cm .4cm clip=true, 
    totalheight=0.28\textheight]{Figures/c/bwrcore/44/2/rf_plots/partial/%
        partial_energy_basis_comparison_fission-44}
    \caption{Relative error for the 44-group, BWR-Core 2 test problem using 
        snapshots of both $\phi$ and $J_{\text{left}}$}
    \label{fig:BWR2_combined}
\end{figure*}

\subsubsection{238-Group Results}

\FIGURE{fig:BWR2_phi-238} shows the results from using snapshots of only the 
$\phi$ for the 238-group cross-section library.  The impractical model of 
Full-Core requires approximately order 22 to reach the goal, while the 
Combined-Assemblies case requires approximately order 31.  Each of these models 
also 
reach the second goal of an order of magnitude reduction in the required energy 
degrees of freedom.  The Combined-Pins model does not reach the error threshold 
within the order goal.  

\begin{figure*}[tb]
    \centering
    \begin{subfigure}{0.5\textwidth}
        \centering
    \includegraphics[trim=.1cm .25cm 2.0cm .4cm clip=true, 
    totalheight=0.238\textheight]{Figures/c/bwrcore/238/2/rf_plots/phi/%
        energy_basis_comparison_fission-44}
    \end{subfigure}%
    \begin{subfigure}{0.5\textwidth}
        \centering
    \includegraphics[trim=.1cm .25cm 2.0cm .4cm clip=true, 
    totalheight=0.238\textheight]{Figures/c/bwrcore/238/2/rf_plots/phi/%
        energy_basis_comparison_fission-238}
    \end{subfigure}
    \caption{Relative error for 238-group, BWR-Core 2 test problem using 
        snapshots of only $\phi$}
    \label{fig:BWR2_phi-238}
\end{figure*}

The results for using snapshots of $J_{\text{left}}$ for basis 
generation are shown in \FIG{fig:BWR2_partial-238}.  The Full-Core model 
requires approximately order 10 to reach the goal, while the 
Combined-Assemblies 
model requires approximately order 36.  The performance of both of 
the Combined models are worse than for the case of $\phi$ snapshots.

\begin{figure*}[tb]
    \centering
    \begin{subfigure}{0.5\textwidth}
        \centering
    \includegraphics[trim=.1cm .25cm 2.0cm .4cm clip=true, 
    totalheight=0.238\textheight]{Figures/s/bwrcore/238/2/rf_plots/partial/%
        partial_energy_basis_comparison_fission-44}
    \end{subfigure}%
    \begin{subfigure}{0.5\textwidth}
        \centering
    \includegraphics[trim=.1cm .25cm 2.0cm .4cm clip=true, 
    totalheight=0.238\textheight]{Figures/s/bwrcore/238/2/rf_plots/partial/%
        partial_energy_basis_comparison_fission-238}
    \end{subfigure}
    \caption{Relative error for 238-group, BWR-Core 2 test problem using 
        snapshots of only $J_{\text{left}}$}
    \label{fig:BWR2_partial-238}
\end{figure*}

Again, the best results are obtained by combining the snapshots of the 
$\phi$ and $J_{\text{left}}$ together as \FIG{fig:BWR2_combined-238} shows.  The required order 
for the full core model is 
about 9, and the Combined-Assemblies model requires approximately order 27 to 
reach the error goal.  The Combined-Pins model is greatly improved and reaches 
the error goal at about order 27.  Clearly from this section, it is difficult 
to achieve an order of magnitude reduction in the required energy degrees of 
freedom and still reach the error goal for configuration 2.  The inhomogeneity 
simply requires more information to model accurately.

\begin{figure*}[tb]
    \centering
    \begin{subfigure}{0.5\textwidth}
        \centering
    \includegraphics[trim=.1cm .25cm 2.0cm .4cm clip=true, 
    totalheight=0.238\textheight]{Figures/c/bwrcore/238/2/rf_plots/partial/%
        partial_energy_basis_comparison_fission-44}
    \end{subfigure}%
    \begin{subfigure}{0.5\textwidth}
        \centering
    \includegraphics[trim=.1cm .25cm 2.0cm .4cm clip=true, 
    totalheight=0.238\textheight]{Figures/c/bwrcore/238/2/rf_plots/partial/%
        partial_energy_basis_comparison_fission-238}
    \end{subfigure}
    \caption{Relative error for 238-group, BWR-Core 2 test problem using 
        snapshots of both $\phi$ and $J_{\text{left}}$}
    \label{fig:BWR2_combined-238}
\end{figure*}

\section{Higher-Order Angular Moments}

After considering the effect of $\phi$ and $J_{\text{left}}$ snapshots, 
the next step is to compare snapshots of higher-order angular moments.  These 
snapshots are generated by an angular expansion by Jacobi basis functions of 
the angular flux.  In 
this case, the zeroth order is exactly $J_{\text{left}}$, while the 
higher moments do not have direct physical corollaries.  These snapshots were 
used for each test problem for basis generation.

\subsection{10-pin Problem}

In this section,  The performance of using additional moments is compared for 
only the Full 10-pin model, as the resulting conclusions are the same for the 
excluded figures.  Each of the figures in this 
section are presented with only a single snapshot model that uses various 
schemes to combine the snapshots as described for each figure.

\subsubsection{44-Group Results}

The individual performance of each set of snapshots for the 44-group 10-pin 
test problem are presented in \FIG{fig:10-pin_10-pin-single}.  The 
performance of $\phi$ and $J_{\text{left}}$ (0th moment) snapshots are 
identical to the results presented previously.  As shown, the higher-order 
moment snapshots do not perform well, and require at least 30th order to reach 
the error threshold.

\begin{figure*}[tb]
    \centering
    \includegraphics[trim=.1cm .25cm 2.0cm .4cm clip=true, 
    totalheight=0.28\textheight]{Figures/s/10-pin/44/rf_plots/angular/%
        angular_comparison_fission_10-pin-44}
    \caption{Relative error for 44-group, 10-pin test problem using 
        snapshots from the 10-pin model.  Sets of snapshots are used 
        separately for basis generation}
    \label{fig:10-pin_10-pin-single}
\end{figure*}

When used together with the snapshots of $\phi$ and $J_{\text{left}}$, the 
results are presented in \FIG{fig:10-pin_10-pin-combined}.  The 
performance of the higher-order moments are nearly identical to that of 
combining only $\phi$ and $J_{\text{left}}$, which is labeled as 0th moment in 
the figure.

\begin{figure*}[tb]
    \centering
    \includegraphics[trim=.1cm .25cm 2.0cm .4cm clip=true, 
    totalheight=0.28\textheight]{Figures/c/10-pin/44/rf_plots/angular/%
        angular_comparison_fission_10-pin-44}
    \caption{Relative error for 44-group, 10-pin test problem using 
        snapshots from the 10-pin model.  Sets of snapshots are combined 
        together for basis generation}
    \label{fig:10-pin_10-pin-combined}
\end{figure*}

\subsubsection{238-Group Results}

This section presents the results of using the higher-order moment snapshots 
for the 238-group 10-pin test problem.  All of the figures in 
this section are shown with side-by-side comparisons of the same data, one 
shown to 43th order, while the other presents the full spectrum up to 
237$^{th}$ order.  \FIGURE{fig:10-pin_10-pin-single-238} presents the results 
from using the various sets of snapshots separately.  Similarly to the 44-group 
results, the higher-order moment snapshots do not perform as well as the 
results of $\phi$ and $J_{\text{left}}$ snapshots.

\begin{figure*}[tb]
    \centering
    \begin{subfigure}{0.5\textwidth}
        \centering
        \includegraphics[trim=.1cm .25cm 2.0cm .4cm clip=true, 
        totalheight=0.238\textheight]{Figures/s/10-pin/238/rf_plots/angular/%
            angular_comparison_fission_10-pin-44}
    \end{subfigure}%
    \begin{subfigure}{0.5\textwidth}
        \centering
        \includegraphics[trim=.1cm .25cm 2.0cm .4cm clip=true, 
        totalheight=0.238\textheight]{Figures/s/10-pin/238/rf_plots/angular/%
            angular_comparison_fission_10-pin-238}
    \end{subfigure}
    \caption{Relative error for 238-group, 10-pin test problem using 
        snapshots from the 10-pin model.  Sets of snapshots are used 
        separately for basis generation}
    \label{fig:10-pin_10-pin-single-238}
\end{figure*}

\FIGURE{fig:10-pin_10-pin-combined-238} presents the results of combining the 
various sets of snapshots.  The success of the snapshots of higher-order 
moments do not significantly change the results from combining only the 
snapshots of $\phi$ and $J_{\text{left}}$.

\begin{figure*}[tb]
    \centering
    \begin{subfigure}{0.5\textwidth}
        \centering
        \includegraphics[trim=.1cm .25cm 2.0cm .4cm clip=true, 
        totalheight=0.238\textheight]{Figures/c/10-pin/238/rf_plots/angular/%
            angular_comparison_fission_10-pin-44}
    \end{subfigure}%
    \begin{subfigure}{0.5\textwidth}
        \centering
        \includegraphics[trim=.1cm .25cm 2.0cm .4cm clip=true, 
        totalheight=0.238\textheight]{Figures/c/10-pin/238/rf_plots/angular/%
            angular_comparison_fission_10-pin-238}
    \end{subfigure}
    \caption{Relative error for 238-group, 10-pin test problem using 
        snapshots from the 10-pin model.  Sets of snapshots are combined 
        together for basis generation}
    \label{fig:10-pin_10-pin-combined-238}
\end{figure*}

\section{BWR Test Problem}

In this section,  The performance of using additional moments is compared for 
only the Full-Core model.  Each of the figures in this 
section are presented with only a single snapshot model that uses various 
schemes to combine the snapshots as described for each figure.  The performance 
of the higher-order moment snapshots are compared between each of the three 
core configurations as discussed previously.

\subsection{Configuration 0}

\subsubsection{44-Group Results}

The individual performance of each set of snapshots for the 44-group BWR-Core 0
test problem are presented in \FIG{fig:BWR0-core-single}.  The 
performance of $\phi$ and $J_{\text{left}}$ (0th moment) snapshots are 
identical to the results presented previously.  As shown, the higher-order 
moment snapshots perform similarly to snapshots of $\phi$.

\begin{figure*}[tb]
    \centering
    \includegraphics[trim=.1cm .25cm 2.0cm .4cm clip=true, 
    totalheight=0.28\textheight]{Figures/s/bwrcore/44/0/rf_plots/angular/%
        angular_comparison_fission_core-44}
    \caption{Relative error for 44-group, BWR-Core 0 test problem using 
        snapshots from the Full-Core model.  Sets of snapshots are 
        used separately for basis generation}
    \label{fig:BWR0-core-single}
\end{figure*}

When used together with the snapshots of $\phi$ and $J_{\text{left}}$, the 
results are presented in \FIG{fig:BWR0-core-combined}.  The 
performance of the higher-order moments are nearly identical to that of 
combining only $\phi$ and $J_{\text{left}}$, which is labeled as 0th moment in 
the figure.

\begin{figure*}[tb]
    \centering
    \includegraphics[trim=.1cm .25cm 2.0cm .4cm clip=true, 
    totalheight=0.28\textheight]{Figures/c/bwrcore/44/0/rf_plots/angular/%
        angular_comparison_fission_core-44}
    \caption{Relative error for 44-group, BWR-Core 0 test problem using 
        snapshots from the Full-Core model.  Sets of snapshots are combined 
        together for basis generation}
    \label{fig:BWR0-core-combined}
\end{figure*}

\subsubsection{238-Group Results}

This section presents the results of using the higher-order moment snapshots 
for the 238-group BWR-Core 0 test problem.  All of the figures in 
this section are shown with side-by-side comparisons of the same data, one 
shown to 43th order, while the other presents the full spectrum up to 
237$^{th}$ order.  \FIGURE{fig:BWR0-core-single-238} presents the results 
from using the various sets of snapshots separately.

\begin{figure*}[tb]
    \centering
    \begin{subfigure}{0.5\textwidth}
        \centering
        \includegraphics[trim=.1cm .25cm 2.0cm .4cm clip=true, 
        totalheight=0.238\textheight]{Figures/s/bwrcore/238/0/rf_plots/angular/%
            angular_comparison_fission_core-44}
    \end{subfigure}%
    \begin{subfigure}{0.5\textwidth}
        \centering
        \includegraphics[trim=.1cm .25cm 2.0cm .4cm clip=true, 
        totalheight=0.238\textheight]{Figures/s/bwrcore/238/0/rf_plots/angular/%
            angular_comparison_fission_core-238}
    \end{subfigure}
    \caption{Relative error for 238-group, BWR-Core 0 test problem using 
        snapshots from the Full-Core model.  Sets of snapshots are 
        used separately for basis generation}
    \label{fig:BWR0-core-single-238}
\end{figure*}

\FIGURE{fig:BWR0-core-combined-238} presents the results of combining the 
various sets of snapshots.  The success of the snapshots of higher-order 
moments do not significantly change the results from combining only the 
snapshots of $\phi$ and $J_{\text{left}}$.

\begin{figure*}[tb]
    \centering
    \begin{subfigure}{0.5\textwidth}
        \centering
        \includegraphics[trim=.1cm .25cm 2.0cm .4cm clip=true, 
        totalheight=0.238\textheight]{Figures/c/bwrcore/238/0/rf_plots/angular/%
            angular_comparison_fission_core-44}
    \end{subfigure}%
    \begin{subfigure}{0.5\textwidth}
        \centering
        \includegraphics[trim=.1cm .25cm 2.0cm .4cm clip=true, 
        totalheight=0.238\textheight]{Figures/c/bwrcore/238/0/rf_plots/angular/%
            angular_comparison_fission_core-238}
    \end{subfigure}
    \caption{Relative error for 238-group, BWR-Core 0 test problem using 
        snapshots from the Full-Core model.  Sets of snapshots are combined 
        together for basis generation}
    \label{fig:BWR0-core-combined-238}
\end{figure*}

\subsection{Configuration 1}

\subsubsection{44-Group Results}

The individual performance of each set of snapshots for the 44-group BWR-Core 0
test problem are presented in \FIG{fig:BWR1-core-single}.  The 
performance of $\phi$ and $J_{\text{left}}$ snapshots are 
identical to the results presented previously.  As shown, the higher-order 
moment snapshots perform similarly to snapshots of $\phi$.

\begin{figure*}[tb]
    \centering
    \includegraphics[trim=.1cm .25cm 2.0cm .4cm clip=true, 
    totalheight=0.28\textheight]{Figures/s/bwrcore/44/1/rf_plots/angular/%
        angular_comparison_fission_core-44}
    \caption{Relative error for 44-group, BWR-Core 1 test problem using 
        snapshots from the Full-Core model.  Sets of snapshots are 
        used separately for basis generation}
    \label{fig:BWR1-core-single}
\end{figure*}

When used together with the snapshots of $\phi$ and $J_{\text{left}}$, the 
results are presented in \FIG{fig:BWR1-core-combined}.  The 
performance of the higher-order moments are nearly identical to that of 
combining only $\phi$ and $J_{\text{left}}$, which is labeled as 0th moment in 
the figure.

\begin{figure*}[tb]
    \centering
    \includegraphics[trim=.1cm .25cm 2.0cm .4cm clip=true, 
    totalheight=0.28\textheight]{Figures/c/bwrcore/44/1/rf_plots/angular/%
        angular_comparison_fission_core-44}
    \caption{Relative error for 44-group, BWR-Core 1 test problem using 
        snapshots from the Full-Core model.  Sets of snapshots are combined 
        together for basis generation}
    \label{fig:BWR1-core-combined}
\end{figure*}

\subsubsection{238-Group Results}

This section presents the results of using the higher-order moment snapshots 
for the 238-group BWR-Core 0 test problem.  All of the figures in 
this section are shown with side-by-side comparisons of the same data, one 
shown to 43th order, while the other presents the full spectrum up to 
237$^{th}$ order.  \FIGURE{fig:BWR1-core-single-238} presents the results 
from using the various sets of snapshots separately.

\begin{figure*}[tb]
    \centering
    \begin{subfigure}{0.5\textwidth}
        \centering
        \includegraphics[trim=.1cm .25cm 2.0cm .4cm clip=true, 
        totalheight=0.238\textheight]{Figures/s/bwrcore/238/1/rf_plots/angular/%
            angular_comparison_fission_core-44}
    \end{subfigure}%
    \begin{subfigure}{0.5\textwidth}
        \centering
        \includegraphics[trim=.1cm .25cm 2.0cm .4cm clip=true, 
        totalheight=0.238\textheight]{Figures/s/bwrcore/238/1/rf_plots/angular/%
            angular_comparison_fission_core-238}
    \end{subfigure}
    \caption{Relative error for 238-group, BWR-Core 1 test problem using 
        snapshots from the Full-Core model.  Sets of snapshots are 
        used separately for basis generation}
    \label{fig:BWR1-core-single-238}
\end{figure*}

\FIGURE{fig:BWR1-core-combined-238} presents the results of combining the 
various sets of snapshots.  The success of the snapshots of higher-order 
moments do not significantly change the results from combining only the 
snapshots of $\phi$ and $J_{\text{left}}$.

\begin{figure*}[tb]
    \centering
    \begin{subfigure}{0.5\textwidth}
        \centering
        \includegraphics[trim=.1cm .25cm 2.0cm .4cm clip=true, 
        totalheight=0.238\textheight]{Figures/c/bwrcore/238/1/rf_plots/angular/%
            angular_comparison_fission_core-44}
    \end{subfigure}%
    \begin{subfigure}{0.5\textwidth}
        \centering
        \includegraphics[trim=.1cm .25cm 2.0cm .4cm clip=true, 
        totalheight=0.238\textheight]{Figures/c/bwrcore/238/1/rf_plots/angular/%
            angular_comparison_fission_core-238}
    \end{subfigure}
    \caption{Relative error for 238-group, BWR-Core 1 test problem using 
        snapshots from the Full-Core model.  Sets of snapshots are combined 
        together for basis generation}
    \label{fig:BWR1-core-combined-238}
\end{figure*}

\subsection{Configuration 2}

\subsubsection{44-Group Results}

The individual performance of each set of snapshots for the 44-group BWR-Core 0
test problem are presented in \FIG{fig:BWR2-core-single}.  The 
performance of $\phi$ and $J_{\text{left}}$ snapshots are 
identical to the results presented previously.  As shown, the higher-order 
moment snapshots perform similarly to snapshots of $\phi$.

\begin{figure*}[tb]
    \centering
    \includegraphics[trim=.1cm .25cm 2.0cm .4cm clip=true, 
    totalheight=0.28\textheight]{Figures/s/bwrcore/44/2/rf_plots/angular/%
        angular_comparison_fission_core-44}
    \caption{Relative error for 44-group, BWR-Core 2 test problem using 
        snapshots from the Full-Core model.  Sets of snapshots are 
        used separately for basis generation}
    \label{fig:BWR2-core-single}
\end{figure*}

When used together with the snapshots of $\phi$ and $J_{\text{left}}$, the 
results are presented in \FIG{fig:BWR2-core-combined}.  The 
performance of the higher-order moments are nearly identical to that of 
combining only $\phi$ and $J_{\text{left}}$, which is labeled as 0th moment in 
the figure.

\begin{figure*}[tb]
    \centering
    \includegraphics[trim=.1cm .25cm 2.0cm .4cm clip=true, 
    totalheight=0.28\textheight]{Figures/c/bwrcore/44/2/rf_plots/angular/%
        angular_comparison_fission_core-44}
    \caption{Relative error for 44-group, BWR-Core 2 test problem using 
        snapshots from the Full-Core model.  Sets of snapshots are combined 
        together for basis generation}
    \label{fig:BWR2-core-combined}
\end{figure*}

\subsubsection{238-Group Results}

This section presents the results of using the higher-order moment snapshots 
for the 238-group BWR-Core 0 test problem.  All of the figures in 
this section are shown with side-by-side comparisons of the same data, one 
shown to 43th order, while the other presents the full spectrum up to 
237$^{th}$ order.  \FIGURE{fig:BWR2-core-single-238} presents the results 
from using the various sets of snapshots separately.

\begin{figure*}[tb]
    \centering
    \begin{subfigure}{0.5\textwidth}
        \centering
        \includegraphics[trim=.1cm .25cm 2.0cm .4cm clip=true, 
        totalheight=0.238\textheight]{Figures/s/bwrcore/238/2/rf_plots/angular/%
            angular_comparison_fission_core-44}
    \end{subfigure}%
    \begin{subfigure}{0.5\textwidth}
        \centering
        \includegraphics[trim=.1cm .25cm 2.0cm .4cm clip=true, 
        totalheight=0.238\textheight]{Figures/s/bwrcore/238/2/rf_plots/angular/%
            angular_comparison_fission_core-238}
    \end{subfigure}
    \caption{Relative error for 238-group, BWR-Core 2 test problem using 
        snapshots from the Full-Core model.  Sets of snapshots are 
        used separately for basis generation}
    \label{fig:BWR2-core-single-238}
\end{figure*}

\FIGURE{fig:BWR2-core-combined-238} presents the results of combining the 
various sets of snapshots.  The success of the snapshots of higher-order 
moments do not significantly change the results from combining only the 
snapshots of $\phi$ and $J_{\text{left}}$.

\begin{figure*}[tb]
    \centering
    \begin{subfigure}{0.5\textwidth}
        \centering
        \includegraphics[trim=.1cm .25cm 2.0cm .4cm clip=true, 
        totalheight=0.238\textheight]{Figures/c/bwrcore/238/2/rf_plots/angular/%
            angular_comparison_fission_core-44}
    \end{subfigure}%
    \begin{subfigure}{0.5\textwidth}
        \centering
        \includegraphics[trim=.1cm .25cm 2.0cm .4cm clip=true, 
        totalheight=0.238\textheight]{Figures/c/bwrcore/238/2/rf_plots/angular/%
            angular_comparison_fission_core-238}
    \end{subfigure}
    \caption{Relative error for 238-group, BWR-Core 2 test problem using 
        snapshots from the Full-Core model.  Sets of snapshots are combined 
        together for basis generation}
    \label{fig:BWR2-core-combined-238}
\end{figure*}

\section{Conclusion}

As the 1-D test problems show, KLT basis sets are highly effective.  Nearly all the of the KLT 
basis sets outperform the mDLP and DLP basis sets, which is expected due to the amount of 
information captured by the KLT.  In nearly all cases, the best performing KLT basis functions are 
those based on snapshots of $\phi$ combined with snapshots of $J_{\text{left}}$.  

It is apparent that including additional information in the snapshots will improve the results only 
if the new information is distinct from the current set of snapshots.  The higher-order moments did 
not improve the overall performance of the basis set because the information contained in the 
higher-order moments was already captured in the 0th moments ($J_{\text{left}}$).

Furthermore, successful KLT basis functions contain all available material types.  As the results 
from the 10-pin test problem show, disregarding snapshots from a fuel type leads to poor expansions 
of the actual solution.  

Finally, as expected, more difficult problems require a larger number of basis functions for an 
expansion of the same relative error.  As the difficulty of the BWR models increases, the number of 
required degrees of freedom to meet the goal were also increased.  Thus, the success of the KLT is 
problem dependent.