% Chapter Template

\chapter{Conclusions and Future Work} % Main chapter title

\label{Chapter7}

\lhead{Chapter 7. \emph{Conclusions}}


\section{Summary}

The overarching goal of this work was to find an effective basis 
with which to expand the energy variable for the eigenvalue response matrix 
method such that the required energy degrees of freedom are reduced by an order 
of magnitude while achieving sub-$0.1\%$ relative error in the pin powers or 
fission densities,  To this end, the Karhunen Lo\'{e}ve Transform was explored 
to create the set of vectors for the energy variable.  The KLT basis was 
compared to the discrete Legendre polynomials and the modified DLPs.  

To base the comparison, three test problems were constructed including a (1) 1-D 
10-pin model representative of a UO$_2$-MOX junction, (2) a 1-D, 70-pin model 
representative of a BWR core, and (3) a 2-D model based on the C5G7 benchmark 
expanded to 44 energy groups.  The difficulty of the models increased in the 
order of presentation, thus the 10-pin test problem was the simplest and the 
C5G7 benchmark was the most difficult.  This difficulty manifested in an 
increase in the number of required energy degrees of freedom necessary to 
achieve sub-$0.1\%$ maximum errors relative to a full multigroup solution of 
the particular test problem.

In nearly all cases, the KLT basis outperformed the mDLP and DLP bases for ERMM 
energy expansions.  This result is not surprising due to the method of 
constructing the KLT basis.  The KLT identifies the most common characteristics 
of a set of functions (or snapshots in the case of this work) and creates the 
basis functions from those characteristics. KLT is ordered such that the most 
important information of a problem-space is contained in the low-order 
basis vectors, thus in general the KLT can achieve high-fidelity expansions 
with few orders.  The downside to KLT is the requirement of initial data with 
which to create the basis vectors.  Since the actual solution is deemed 
unavailable (because requiring the solution to approximate the solution is a 
waste of effort for this application), a number of representative snapshot 
models were constructed for each of the test problems.  

The choice of snapshot model greatly impacts the effectiveness of a KLT basis, 
and thus the results of an expansion in that basis.  In general, snapshot 
models should be computationally quick to solve, yet similar to the test 
problem.  The results presented in the body of this work indicate that to be 
effective, a snapshot model does not need to be a reduced version of the test 
problem, but the snapshot model should contain features consistent with the 
test problem.  

For example, consider the results of the 10-pin snapshot 
models.  The results of the UO$_2$ and MOX models were not desirable as each of 
the models was lacking the information of one of the fuel types of the 10-pin 
problem; however, the results were greatly improved when the snapshots of those 
two models were included together with the snapshots representing the 
UO$_2$-MOX pin junction. This trend remains consistent throughout the 
remainder of the test problems.  The small snapshot models can be as 
effective as the larger models provided that the unique features of a test 
problem are captured by the snapshot model.  

In regards to the overarching goal, the KLT can be used to construct an 
effective basis set for expansion in ERMM, and the goal of an order of 
magnitude reduction in the required energy degrees of freedom can be achieved 
in some cases.  Unfortunately, those cases typically utilized the solution of 
the test-problem as snapshots for basis generation, which is not a realistic 
use of the KLT basis for this application.  However, the KLT basis functions 
utilizing the test problem represent the best approximation of a truncated 
basis expansion for a given order.  In other words, there is not another basis 
set that can outperform a basis set created by KLT using snapshots of the test 
problem.  

Many of the smaller snapshot models presented throughout this work provided 
encouraging results due to the simplicity of the snapshot model, particularly 
the 1-D approximation to the 2-D test problem.  Such a method could be 
used to greatly improve the computation time of response matrix methods.  
As mentioned previously, response matrix solutions are very slow compared 
to other solution methods e.g., discrete ordinance unless the phase space 
is significantly reduced.  Using KLT for energy can provide a greater reduction 
in the the required energy degrees of freedom as compared to the modified DLPs 
provided that the snapshot model(s) are chosen well. 

The results also indicated that including additional information in the form of 
snapshots will typically improve the results as long as the extra information 
is relevant to the expansion.  For example, the scalar flux expansion was 
improved by including the leftward partial current; however, the results 
remained relatively unchanged if additional higher order angular moments were 
also included in the basis generation.  This suggests that adding more snapshots 
improves the data only if the new snapshots are unique and relevant.  Therefore,
the inclusion of partial current snapshots improved results because the 
snapshots differed meaningfully from scalar flux snapshots.
Contrarily, the use of a greater number of snapshots from a finer spatial mesh 
yielded diminishing improvement because the new 
snapshots became increasingly similar.

The final conclusion to make is when considering test problems using different 
energy groupings (i.e., 44-group and 238-group), the resulting relative fission 
density errors were approximately the same between the two libraries, suggesting 
that KLT efficiency is almost independent of the number of groups in the 
library.  KLT can capture a large amount of physics information in 
low-order vectors, and, thus, KLT can perform as well or better with the 
inclusion of additional information from a higher group library compared to a 
lower group library.  

\section{Future Work}

The effort in expanding the energy variable has provided some insight, but 
ultimately more work is required to finish the exploration of the KLT as a 
basis for expanding the energy variable.  There are a few direct extensions of 
work presented here that would be of substantial value to this area of 
research.  

The first such project is to further explore the KLT as applied to 2-D 
problems.  Much of the 2-D work presented here was truncated due to a time 
requirement.  The C5G7 results would benefit from computations at higher 
spatial and angular order to allow for better congruence between the ERMM 
solution and other accepted solutions for the C5G7 benchmark.  

In addition, it would be of substantial value to explore other energy grouping 
to determine a dependence of KLT on the number of energy groups in an 
expansion.  Although the results here suggested a weak dependence, further 
research 
is needed to confirm these findings.  

Any ongoing work with KLT will additionally focus on snapshot generation and 
effectiveness.  Most problems of interest contain a 
sufficiently large number of unique spatial
regions for use in snapshot generation; therefore, a key goal of ongoing work is 
to determine other variables (e.g, increased angle dependence)
that can provide greater snapshot variety.  

Another interesting test of KLT performance would be to use the each of the KLT 
basis sets to expand the SERMENT solution vectors directly, instead of 
solving for the pin powers or fission densities.  This would be interesting to 
ensure that the behavior is similar to the previously presented results.  It is 
assumed that the general trend will be similar, but the nonlinear behavior 
would likely differ as a function of expansion order as compared to the results 
presented here.

Finally, this work has focused solely on the expansion of the energy 
variable. However, ERMM requires expansion in all phase-space 
variables, and it would be interesting to compare the effectiveness 
of a KLT expansion for either the spatial or angular variable.  
Provided that snapshots can be obtained, the KLT is not specific to 
any particular variable, thus is expected to provide an effective 
expansion in any phase-space variable to which it is applied.  
