% Appendix A

\chapter{Parametric Studies} % Main appendix title

\label{AppendixA} % For referencing this appendix elsewhere, use \ref{AppendixA}

\lhead{Appendix A. \emph{Parametric Studies}} 

Over the course of the project, several parametric studies were devised and 
implemented to ensure reasonable choices for each of the parameters discussed 
in this Appendix.  The parametric studies may be grouped into two sections: 
snapshot parameters, and database parameters.
  
\section{Snapshot Parameters}

\subsection{Sensitivity to Snapshot Selection}

The first snapshot study was designed to determined which snapshots created 
the most effective basis sets.  For this study, a variable number of snapshots 
from the baseline BWR test problem were chosen.  These snapshots were then 
used to generate KLT basis sets, which were used to reconstruct the pin powers 
for the BWR test-problem.  The available snapshots totaled 1960 (28 
snapshots from each of the 70 pins), however only half of the snapshots were 
unique due to the problem symmetry.  

The study used several snapshot selection schemes to 
select from the total snapshots.  The first scheme was to use evenly spaced 
snapshots (i.e., only every $x$th snapshot was used to generate the basis).  The 
second used a 
random selection of a given number.  The third used
the first $x$ snapshots, and the last scheme used $x$ snapshots from the 
center (because the problem was spatially symmetric). 

\begin{figure*}[tb]
    \centering
    \begin{subfigure}{0.5\textwidth}
        \centering
        \includegraphics[trim=.1cm 1.0cm 1.7cm .4cm clip=true, 
        totalheight=0.23\textheight]{Figures/parametric/%
            snapcase0_energy_basis_comparison_fission}
        \caption{Using evenly spaced snapshots}
        \label{fig:snapcase0}
    \end{subfigure}%
    \begin{subfigure}{0.5\textwidth}
        \centering
        \includegraphics[trim=.1cm 1.0cm 1.7cm .4cm clip=true, 
        totalheight=0.23\textheight]{Figures/parametric/%
            snapcase1_energy_basis_comparison_fission}
        \caption{Using randomly selected snapshots}
        \label{fig:snapcase1}
    \end{subfigure}%
    
    \vspace{0.15cm}
    \begin{subfigure}{0.5\textwidth}
        \centering
        \includegraphics[trim=.1cm 1.0cm 1.7cm .4cm clip=true, 
        totalheight=0.23\textheight]{Figures/parametric/%
            snapcase2_energy_basis_comparison_fission}
        \caption{Using beginning snapshots}
        \label{fig:snapcase2}
    \end{subfigure}%
    \begin{subfigure}{0.5\textwidth}
        \centering
        \includegraphics[trim=.1cm 1.0cm 1.7cm .4cm clip=true, 
        totalheight=0.23\textheight]{Figures/parametric/%
            snapcase3_energy_basis_comparison_fission}
        \caption{Using middle snapshots}
        \label{fig:snapcase3}
    \end{subfigure}%
    \caption{Relative error for 238-group, BWR test problem when using various 
snapshot selection schemes.  The schemes converge as the number of snapshots 
increases. The legend numbers correspond to the number of snapshots selected 
for KLT basis generation.}
\end{figure*}

As shown in Figs. \ref{fig:snapcase0} and \ref{fig:snapcase1}, the resulting 
impact to fission density errors was negligible for the first two 
schemes; however, the last two schemes produced a greater variation until 
snapshots from all types of cells were included as shown in Figs. 
\ref{fig:snapcase2} and \ref{fig:snapcase3}.  In other words, inclusion of 
additional snapshots did not improve the results unless 
new snapshots were meaningfully distinct from the current set of snapshots.  
This study suggests that including all the the available snapshots for a given 
model will provide the best basis sets, at a slight cost of computation time 
generating a basis using additional data.

\subsection{Sensitivity to Fineness of Spatial Mesh}

To determine the effect of adjusting the spatial mesh to produce various 
numbers of snapshots per pincell, a second parametric study was created.  
This study used the 44-group, 10-pin test problem.  The spatial 
discretization was varied to produce various numbers of spatial 
cells and potential snapshots. As described in Chapter \ref{Chapter4}, 
the 10-pin test problem originally had 280 spatial cells (28 per 
pincell).  By proportionally enlarging the mesh regions in the problem, 
the total number of snapshots may be reduced.  

\begin{figure*}[tb]
    \centering
    \includegraphics[trim=.1cm .25cm 2.0cm .4cm clip=true, 
    totalheight=0.28\textheight]{Figures/parametric/%
        number_basis_comparison_fission}
    \caption{Relative error for 44-group, 10-pin test problem using 
        snapshots from a spatially reduced model.  The legend numbers 
correspond to the total number of available snapshots.  The problem was reduced 
from 280 total snapshots.}
\label{fig:nSnapshots}
\end{figure*}

For this study, all available snapshots were used from the spatially reduced 
results.  Results are shown in Fig. \ref{fig:nSnapshots}, and they
indicated that mesh size had a relatively 
small effect on the efficacy of the generated basis set.  Additional snapshots 
from the same model were exceedingly similar
to the other snapshots, which did not add additional information to the 
expansion.

\section{Database Parameters}

To improve computation time, databases that included responses for every order 
were created for each problem and model, allowing the relevant information to 
be read quickly without using on-the-fly computations.  Responses are a function 
of the $k$ eigenvalue; thus, two studies were designed that explored both the 
number of $k$ values selected as well as the range of the $k$ values.  Both of 
these studies were based on the BWR test problem, configuration 0.  

\subsection{Sensitivity to Number of k Values in Databases}

The first study for the databases explored how many values of $k$ were needed 
for database success.  A cubic interpolation was used in conjunction with the 
database.  Thus, a minimum of three values of $k$ on either side (total of 6 
$k$ values) of the true value of $k$ for the BWR test problem were used in the 
database.  

As shown in Fig. \ref{fig:nK}, adding more values of $k$ to the database as no 
effect on the relative fission density error.  However, adding additional 
values of $k$ to the database significantly increased the computation time of 
the databases.  However, for more difficult problems, the converged value of 
$k$ for a given model may be different from the true value, thus a small buffer 
in the number of $k$ value was chosen by selecting eight values of $k$ in each 
database, centered about the reference $k$ value.

\begin{figure*}[tb]
    \centering
    \includegraphics[trim=.1cm .25cm 2.0cm .4cm clip=true, 
    totalheight=0.28\textheight]{Figures/parametric/%
        nK_energy_basis_comparison_fission}
    \caption{Relative error for 238-group, BWR test problem a database of 
responses filled with a number of $k$ values as indicated by the legend.  The 
values spanned the range of $\pm0.15$ of the the true value of $k$.  The number 
of $k$ values had no effect of the results.}
    \label{fig:nK}
\end{figure*}

\subsection{Sensitivity to Range of k Values in Databases}

The second parametric study for the databases was to determine the necessary 
range for the $k$ values to span.  The range was centered on the reference 
value for $k$ and used a total of eight values of $k$ within the database. The 
results are shown in Fig. \ref{fig:kdel}.  There was no difference in the 
fission density as a function of the spanned range.  Thus a range of $\pm$ 0.15 
from the reference $k$ value was chosen for all test problems and models.  The 
databases were found to not influence the results with the chosen parameters.

\begin{figure*}[tb]
    \centering
    \includegraphics[trim=.1cm .25cm 2.0cm .4cm clip=true, 
    totalheight=0.28\textheight]{Figures/parametric/%
        kdel_energy_basis_comparison_fission}
    \caption{Relative error for 238-group, BWR test problem a database of 
        responses filled with eight $k$ values. The 
        values spanned the range of $\pm$ the value in the legend of the the 
true value of $k$.  The size of the range no effect of the results.}
    \label{fig:kdel}
\end{figure*}
